\documentclass{sigchi}

% Use this command to override the default ACM copyright statement
% (e.g. for preprints).  Consult the conference website for the
% camera-ready copyright statement.

%% HOW TO OVERRIDE THE DEFAULT COPYRIGHT STRIP --
%% Please note you need to make sure the copy for your specific
%% license is used here!
% \toappear{
% Permission to make digital or hard copies of all or part of this work
% for personal or classroom use is granted without fee provided that
% copies are not made or distributed for profit or commercial advantage
% and that copies bear this notice and the full citation on the first
% page. Copyrights for components of this work owned by others than ACM
% must be honored. Abstracting with credit is permitted. To copy
% otherwise, or republish, to post on servers or to redistribute to
% lists, requires prior specific permission and/or a fee. Request
% permissions from \href{mailto:Permissions@acm.org}{Permissions@acm.org}. \\
% \emph{CHI '16},  May 07--12, 2016, San Jose, CA, USA \\
% ACM xxx-x-xxxx-xxxx-x/xx/xx\ldots \$15.00 \\
% DOI: \url{http://dx.doi.org/xx.xxxx/xxxxxxx.xxxxxxx}
% }

% Arabic page numbers for submission.  Remove this line to eliminate
% page numbers for the camera ready copy
% \pagenumbering{arabic}

% Load basic packages
\usepackage{balance}       % to better equalize the last page
\usepackage{graphics}      % for EPS, load graphicx instead 
\usepackage[T1]{fontenc}   % for umlauts and other diaeresis
\usepackage{txfonts}
\usepackage{mathptmx}
\usepackage[pdflang={en-US},pdftex]{hyperref}
\usepackage{color}
\usepackage{booktabs}
\usepackage{textcomp}

% Some optional stuff you might like/need.
\usepackage{microtype}        % Improved Tracking and Kerning
% \usepackage[all]{hypcap}    % Fixes bug in hyperref caption linking
\usepackage{ccicons}          % Cite your images correctly!
% \usepackage[utf8]{inputenc} % for a UTF8 editor only

% If you want to use todo notes, marginpars etc. during creation of
% your draft document, you have to enable the "chi_draft" option for
% the document class. To do this, change the very first line to:
% "\documentclass[chi_draft]{sigchi}". You can then place todo notes
% by using the "\todo{...}"  command. Make sure to disable the draft
% option again before submitting your final document.
\usepackage{todonotes}

% Paper metadata (use plain text, for PDF inclusion and later
% re-using, if desired).  Use \emtpyauthor when submitting for review
% so you remain anonymous.
\def\plaintitle{Presence and pressure sensing in textiles}
\def\plainauthor{A.M.J.M. Schoonen, T.R. Nachtigall}
\def\emptyauthor{}
\def\plainkeywords{Capacitive touch; pressure sensing; e-textile}
\def\plaingeneralterms{Documentation, Standardization}

% llt: Define a global style for URLs, rather that the default one
\makeatletter
\def\url@leostyle{%
  \@ifundefined{selectfont}{
    \def\UrlFont{\sf}
  }{
    \def\UrlFont{\small\bf\ttfamily}
  }}
\makeatother
\urlstyle{leo}

% To make various LaTeX processors do the right thing with page size.
\def\pprw{8.5in}
\def\pprh{11in}
\special{papersize=\pprw,\pprh}
\setlength{\paperwidth}{\pprw}
\setlength{\paperheight}{\pprh}
\setlength{\pdfpagewidth}{\pprw}
\setlength{\pdfpageheight}{\pprh}

% Make sure hyperref comes last of your loaded packages, to give it a
% fighting chance of not being over-written, since its job is to
% redefine many LaTeX commands.
\definecolor{linkColor}{RGB}{6,125,233}
\hypersetup{%
  pdftitle={\plaintitle},
% Use \plainauthor for final version.
%  pdfauthor={\plainauthor},
  pdfauthor={\emptyauthor},
  pdfkeywords={\plainkeywords},
  pdfdisplaydoctitle=true, % For Accessibility
  bookmarksnumbered,
  pdfstartview={FitH},
  colorlinks,
  citecolor=black,
  filecolor=black,
  linkcolor=black,
  urlcolor=linkColor,
  breaklinks=true,
  hypertexnames=false
}

% create a shortcut to typeset table headings
% \newcommand\tabhead[1]{\small\textbf{#1}}

% End of preamble. Here it comes the document.
\begin{document}

\title{\plaintitle}

\numberofauthors{2}
\author{%
  \alignauthor{A.M.J.M. Schoonen\\
    \affaddr{}\\
    \affaddr{Eindhoven, The Netherlands}\\
    \email{admar@familieschoonen.nl}}\\
  \alignauthor{T.R. Nachtigall\\
    \affaddr{Eindhoven University of Technology}\\
    \affaddr{Eindhoven, The Netherlands}\\
    \email{T.R.Nachtigall@tue.nl}}\\
}

\maketitle

\begin{abstract}
  UPDATED---\today. This paper describes a method to add presence sensing to
  pressure sensors, thereby allowing to detect the presence of humans before
  they touch the pressure sensor. This allows for novel interfaces that guide
  users even before they deliberately use and interact with the object. In
  principle, the method only requires a software modification so there are no
  additional costs for materials and the feature could be made available to
  existing products with a software update. This method is used to add distance
  and presense sensing to pressure sensors in textile, using well known
  materials such as Velostat as pressure sensor.
\end{abstract}

\category{H.5.2.}{Information Interfaces and Presentation
  (e.g. HCI)}{Input devices and strategies (e.g., mouse, touchscreen)
} 

\keywords{\plainkeywords}

\section{Introduction}

\section{Application Implications}

\section{Novelty}

\section{Relevancy}

\section{Technical Aspects}
\subsection{Resistive Pressure Sensors}
Low cost pressure sensors are often made of a sandwich of flexible electrodes
with a layer of flexible moderately conductive material in between. The
moderately conductive material is usually made of a carbon impregnated polymer
with a specific structure that allows it to be squeezed together. The material
can be considered as having many parallel resistors. When the material is
compressed some of the resistors will be partially short-circuited due to
non-linear elastic deformations. The partial short-circuits result in a lower
overall resistance of the sandwitch structure. This is visualized
in Figure \ref{fig:pressure_sensor}.

\begin{figure}[!htbp]
\centering
  \includegraphics[width=0.9\columnwidth]{figures/resistive_sensor}
  \caption{Conceptual model of a pressure sensor. Compressing the sensor causes
  partial short-circuits which lowers the overall resistance of the
  structure.}~\label{fig:pressure_sensor}
\end{figure}

This type of sensor is used in commercial products (FIXME: ADD REFERENCES TO
COMMERCIAL PRESSURE SENSORS?) and is also popular in electronic textile products
(FIXME: ADD REFERENCES TO VELOSTAT SENSORS?) since it is easy to fabricate and
integrates well in textile products.

In these products, the sensor is often used in a resistor divider setup where
one electrode is connected directly to an analog input of a microcontroller and
the other electrode is connected directly to ground. To complete the circuit, a
reference resistor is connected to the analog input and supply voltage.

Since some microcontrollers (such as the Atmel ATmega328P in the popular Arduino
UNO board) have internal pull up resistors on the analog inputs, an external
reference resistor is in some cases not even needed. However, the pull up
resistors are fixed value (10 - 50 $\textrm{k}\Omega$) so the sensor must be matched to
this resistor value. Additionally, these pull up resistors have a higher
tolerance so the gain of the sensor can vary from product to product.

\subsection{Capacitive Touch / Distance Sensors}
Capacitive sensors are popular sensors in embedded computing due to their low cost
and capabilities of detecting approaching human body parts, which allows the
object to give feedback to the user even before the user is physically touching
the object. The physics behind sensors that meaure self-capacitance is that one
plate of the capacitor is formed by the sensor and the other plate is formed by
nearby grounded objects. The capacitance is a function of area and distance, as
shown by the well-known parallel plate model

\begin{equation}
C = {{\varepsilon A} \over {d}}
\end{equation}

where $C$ is the capacitance, $\varepsilon$ is the permittivity of the material
between the plates (approximately $8.85418 \cdot 10^{-12} ~ \textrm{F/m}$ for
air), $A$ is the overlapping area of the plates and $d$ is the distance between
the plates.

For many use cases of capacitive touch, the permittivity and area do not change
significantly and thus the capacitiance is a measure for the distance between
the sensor and the body part.

There are many different methods to measure
self-capacitance (FIXME: ADD REFERENCE TO RC-DISCHARGE, CVD, RELAXATION
OSCILLATOR, OTHERS?). 

\subsubsection{R-C Charge Method}
A very popular method in the Do It Yourself (DIY) community is the R-C charge
method. In this method, the sensor is connected to a digital input pin of a
microcontroller and a resistor is connected to the sensor and a digital output
pin of the microcontroller. Toggling the output pin causes the capacitive sensor
to charge or discharge. These charge and discharge times depend on the value of
the resistor and the value of the capacitor formed by the sensor plate and the
human body and the microcontroller determines the capacitance by measuring these
charge and discharge times.

The R-C charge method is very simple and low cost as it requires only an
additional resistor and two General Purpose Input Output pins (GPIO pins) of a
microcontroller, which makes it very attractive to the DIY community.  However,
since it is based on measuring charge and discharge times it is also relatively
slow.

An intrinsic feature of capacitive touch sensors is that the
electric field needs to fringe out of the object to be able to sense the human
body and due to this fringing, the electric field is also easily disturbed by
other electric fields or nearby grounded objects such as 110 / 230 V wires or
devices or metal structures. The slow measurement method of the R-C charge
method makes it more difficult to filter out these disturbances, leading to poor
performance of the sensor and poor experiences of capacitive touch for users of
the objects which use this method.

\subsubsection{CVD Method}
Another well-known method for self-capacitance does not rely on R-C charge times
but instead relies on a microcontroller with a multiplexed Analog to Digital
Converter (ADC), the sample-and-hold capacitor ($\textrm{C}_{\textrm{S\&H}}$)
inside this ADC and the ability of the microcontroller to dynamically
reconfigure its analog input pins to digital output pins. This method is called
Capacitive Voltage Division (CVD).

In this method, no external resistor is required and the sensor plate is
directly connected to an analog input pin. The microcontroller starts a
measurement by configuring this pin a digital output and making this output low,
thereby discharging the sensor. Next, the microcontroller connects the internal
ADC to its supply voltage, which charges $\textrm{C}_{\textrm{S\&H}}$ to a fixed
amount of charge which depends only on the capacitance
$\textrm{C}_{\textrm{S\&H}}$ and the supply voltage.

Afther the sensor pin is discharged and $\textrm{C}_{\textrm{S\&H}}$ is
charged, the sensor pin needs to be reconfigured as analog input and the
multiplexer of the ADC needs to be switched to this input. This will
redistribute the charge on $\textrm{C}_{\textrm{S\&H}}$ over both
$\textrm{C}_{\textrm{S\&H}}$ and the capacitance of the sensor. A larger sensor
capacitance will result in a lower voltage on the sensor and the last step of
this method is to measure this voltage using the usual ADC functions.

Since this method does not rely on R-C discharge times but uses the internal and
usually much faster ADC, more filtering can be applied to the signal to remove
disturbance of other nearby objects and electronic devices. This results in a
superior performance and a better user experience.

\subsection{Using Resistive Pressure Sensors as Capacitive Distance Sensors}
The CVD method connects the sensor directly to analog input of a
microcontroller, similar to the resistive sensor setup. The resistive sensor can
now be used to also measure capacitance by connecting the other electrode of the
sensor to a GPIO pin instead of ground. This setup is shown in Figures
\ref{fig:cap_res_setup_res} and \ref{fig:cap_res_setup_cap}.

Figure \ref{fig:cap_res_setup_res} shows the setup in resistive sensing mode,
which is a standard resitive divider setup which uses an internal pull resistor
as reference resistor and a digital output pin as ground to complete the
circuit.

Figure \ref{fig:cap_res_setup_cap} shows the same circuit but with the GPIO pins
reconfigured for capacitive sensing. In this setup, the internal pull up
resistor is not used and the to electrode of the pressure sensor is only
connected to the analog input. The bottom electrode is connected to a pin that
is configured as digital input, which effectively means that the sensor is
floating. This is exactly the setup that is needed for the CVD method.

\begin{figure}[!htbp]
\centering
  \includegraphics[width=0.9\columnwidth]{figures/cap_res_setup_res}
  \caption{Resistive pressure sensor used in capacitive and resistive setup in
resistive sensing mode. Grey items are internal to the
microcontroller.}~\label{fig:cap_res_setup_res}
\end{figure}

\begin{figure}[!htbp]
\centering
  \includegraphics[width=0.9\columnwidth]{figures/cap_res_setup_cap}
  \caption{Resistive pressure sensor used in capacitive and resistive setup in
capacitive sensing mode. Grey items are internal to the
microcontroller.}~\label{fig:cap_res_setup_cap}
\end{figure}

\section{Software Features}
In many cases both resistive pressure sensing and capacitive presence sensing
applications require relative measurements but do not depend on absolute
measurements. In such cases, a state machine which tracks any background
variations on the signal level is a simple and effective method to reduce noise.
In our case, both the resistive sensor signals and the capacitive sensor signals
use the following state machine where each signal has its own instance and
its own parameter settings.

The state machine for each sensor can be in five states:
\begin{itemize}
\item Calibrating
\item Released
\item Released to Pressed
\item Pressed
\item Pressed to Released
\item Released
\end{itemize}
This is shown in Figure \ref{fig:state_machine}.

\begin{figure}[!htbp]
\centering
  \includegraphics[width=0.9\columnwidth]{figures/state_machine}
  \caption{State machine for resistive and capacitive measurements to track
background variations.}~\label{fig:state_machine}
\end{figure}

In the Calibrating and Released states, the background variations are tracked
using an exponential decaying filter and stored in average $a$. In all other
states the background variations are not tracked. In the Released state, if the
most recent measurement $x$ is more than $P$ counts below average $a$, the state
is changed to Released to Pressed. If in the next measurement $x$ is less
than $P$ counts below the average, the state is changed back to Released. If
however the next $C$ measurements are all more than $P$ counts below this
average, the state is changed to Pressed.

Similarly the state moves from the Pressed state to the Pressed to Released
state and from Pressed to Released to the Released state.

To account for stuck buttons (for example: when the user unintentionally placed
a large conductive object very close to the capacitive touch sensor while the
sensor was in Released state), there is a maximum time that the sensor can be in
the Pressed state. If after this time the sensor is still in the Pressed state,
it is changed to the Calibrating state and the sensor will start recalibrating.

By changing the parameters $N$, $C$, $P$ and $R$ the amount of filtering and
speed of detection can be tuned to the application.

FIXME: ADD REFERENCE TO STATE MACHINE 

\section{Shielding}
In capacitive sensing mode, the bottom electrode of the resistive pressure
sensor is floating. In this mode the pressure sensitive resistive material in
between can be seen effectively as a conductor and thus the whole sensor can be
seen as just a single electrode. The electric field of the sensor will therefore
also fringe all around the sensor, including the bottom side. As the electric
field is also present at the underside of the sensor, the sensor is not only
sensitive for the presence of human body parts above the sensor, but also below.
If the capacitance underneath the sensor is relatively constant, tuning of the
state machine could be sufficient to filter this out and make the sensor only
sensitive to large and / or rapid variations.

However, for some applications such as lose fitting clothing, this might still
not be sufficient. In such cases, the sensor can be made less sensitive by
adding a shield underneath the sensor. Connecting this shield to ground
effectively removes all of the capacitance variation but also reduces the
sensitivity of the sensor. By connecting the sensor itself to the input of a 1 x
opamp and the output of the opamp to the shield, the voltage of the shield is
always very close to the voltage on the sensor itself. As a result, the electric
field underneath the sensor is virtually zero and no sensitivity is lost. Note
that for proper shielding, also the cable to the sensor should be shielded. A
schematic diagram is shown in Figure \ref{fig:shield_circuit}.

FIXME: ADD REFERENCE TO SHIELDING

\begin{figure}[!htbp]
\centering
  \includegraphics[width=0.9\columnwidth]{figures/shield_circuit}
  \caption{Circuit to shield underside of capacitive
sensor.}~\label{fig:shield_circuit}
\end{figure}

\section{Mechanical Features}
An overview of the stackup of the total sensor is shown in Figure
\ref{fig:stackup}. In this figure, the electrode and shield material can be
conductive textile, the insulating material can be any non-compressable
insulating material (for example cotton) and the pressure sensitive material can
be Velostat or ESD foam.

\begin{figure}[!htbp]
\centering
  \includegraphics[width=0.9\columnwidth]{figures/stackup}
  \caption{Stackup for pressure and presence sensor with
shield}~\label{fig:stackup}
\end{figure}


\section{Conclusion}
In this paper we have described the combination of capacitive distance sensing
using the CVD method for presence sensing with resistive sensing for pressure
sensing. The robustness of the CVD method as well as the required circuit and
microcontroller features make it ideal to combine with existing resistive
pressure sensing applications to enhance the user experience by not only sensing
how hard a user presses on a button but already giving feedback to the user when
he / she is approaching the button.

% BALANCE COLUMNS
\balance{}

% REFERENCES FORMAT
% References must be the same font size as other body text.
\bibliographystyle{SIGCHI-Reference-Format}
\bibliography{sample}

\end{document}

%%% Local Variables:
%%% mode: latex
%%% TeX-master: t
%%% End:

\documentclass{sigchi-ext}
% Please be sure that you have the dependencies (i.e., additional
% LaTeX packages) to compile this example.
\usepackage[T1]{fontenc}
\usepackage{textcomp}
\usepackage[scaled=.92]{helvet} % for proper fonts
\usepackage{graphicx} % for EPS use the graphics package instead
\usepackage{balance}  % for useful for balancing the last columns
\usepackage{booktabs} % for pretty table rules
\usepackage{ccicons}  % for Creative Commons citation icons
\usepackage{ragged2e} % for tighter hyphenation

% Some optional stuff you might like/need.
% \usepackage{marginnote} 
% \usepackage[shortlabels]{enumitem}
% \usepackage{paralist}
% \usepackage[utf8]{inputenc} % for a UTF8 editor only

%% EXAMPLE BEGIN -- HOW TO OVERRIDE THE DEFAULT COPYRIGHT STRIP --
% \copyrightinfo{Permission to make digital or hard copies of all or
% part of this work for personal or classroom use is granted without
% fee provided that copies are not made or distributed for profit or
% commercial advantage and that copies bear this notice and the full
% citation on the first page. Copyrights for components of this work
% owned by others than ACM must be honored. Abstracting with credit is
% permitted. To copy otherwise, or republish, to post on servers or to
% redistribute to lists, requires prior specific permission and/or a
% fee. Request permissions from permissions@acm.org.\\
% {\emph{CHI'14}}, April 26--May 1, 2014, Toronto, Canada. \\
% Copyright \copyright~2014 ACM ISBN/14/04...\$15.00. \\
% DOI string from ACM form confirmation}
%% EXAMPLE END

% Paper metadata (use plain text, for PDF inclusion and later
% re-using, if desired).  Use \emtpyauthor when submitting for review
% so you remain anonymous.
\def\plaintitle{PrePre: Presence and Pressure Sensing}
\def\plainauthor{Troy Nachtigall, A.M.J.M. Schoonen}
\def\emptyauthor{}
\def\plaincategory{H.5.2.Information Interfaces and Presentation, I.4.8 Sensor Fusion}
\def\plainkeywords{Sensor; Sensing; Presence; Pressure; Capacitive; Resistive; Textile; E-textile}
\def\plaingeneralterms{Documentation, Standardization}

\title{PrePre: Presence and Pressure Sensing}

\numberofauthors{2}
% Notice how author names are alternately typesetted to appear ordered
% in 2-column format; i.e., the first 4 autors on the first column and
% the other 4 auhors on the second column. Actually, it's up to you to
% strictly adhere to this author notation.
\author{%
  \alignauthor{%
    \textbf{First Author}\\
  	\affaddr{AuthorCo, Inc.}\\
  	\affaddr{123 Author Ave.}\\
  	\affaddr{Authortown, PA 54321 USA}\\
  	\email{author1@anotherco.com}}
    %\textbf{Troy Nachtigall}\\
    %\affaddr{Eindhoven Univsersity of Technology} \\
    %\affaddr{Eindhoven,  5400MB, NL} \\
    %\email{t.r.nachtigall@tue.nl} }
    \alignauthor{%
    \textbf{Second Author}\\
  	\affaddr{AuthorCo, Inc.}\\
  	\affaddr{123 Author Ave.}\\
  	\affaddr{Authortown, PA 54321 USA}\\
  	\email{author2@anotherco.com}}
    %\textbf{A.M.J.M. Schoonen}\\
    %\affaddr{Eindhoven, The Netherlands}\\
    %\email{admar@familieschoonen.nl} } 
    }


% Make sure hyperref comes last of your loaded packages, to give it a
% fighting chance of not being over-written, since its job is to
% redefine many LaTeX commands.
\begin{marginfigure}
\vspace{1.0mm}
\begin{minipage}{\marginparwidth}
\centering
\includegraphics[width=0.9\columnwidth]{figures/sensor}
 \caption{PrePre sensor prototype.}~\label{fig:sensor}
\end{minipage}
\end{marginfigure}

\definecolor{linkColor}{RGB}{6,125,233}
\hypersetup{%
  pdftitle={\plaintitle},
%  pdfauthor={\plainauthor},
  pdfauthor={\emptyauthor},
  pdfkeywords={\plainkeywords},
  bookmarksnumbered,
  pdfstartview={FitH},
  colorlinks,
  citecolor=black,
  filecolor=black,
  linkcolor=black,
  urlcolor=linkColor,
  breaklinks=true,
}

% \reversemarginpar%

\begin{document}

\maketitle

% Uncomment to disable hyphenation (not recommended)
% https://twitter.com/anjirokhan/status/546046683331973120

\RaggedRight{} 

% Do not change the page size or page settings.

\begin{abstract}
  Designing with properties such as touch sensing, distance sensing and digital control enables new dimensions within fashion and design. A range of new possibilities for sensing, tactility and functionality. Resistive pressure sensing and capacitive presence sensing are not new in wearable technology. However, there is still limited insight into the potential of soft materials capable of performing multiple functions at the same time. Adding multiple functionalities is fundamental to the exploitation of new e-textile properties. Development of multifunctional textiles may pro- vide greater use possibilities for e-textiles where separate components for each sensor were required. 
\end{abstract} 
 
\category{H.5.2}{User Interfaces}{Information Interfaces and Presentation}
\category{H.5.2}{User Interfaces}{Input devices and strategies (e.g., mouse, touchscreen)}
\category{C.3}{Special-Purpose and Application-Based Systems}{Signal processing systems}

\section{Introduction}
There are many challenges facing wearables in terms of the creation of effective
smart wearables \cite{Tomico2015} and embodied services when these are applied to
the body \cite{Bhomer2015}.  

PrePre demonstrates a method to add presence sensing to pressure sensors, allowing to detect the presence of humans before they touch the pressure sensor. This allows for novel interfaces that guide users even before they deliberately use and interact with the object. In principle, the method only requires a software modification so there are no additional costs for materials and the feature could be made available to existing products with a software update. This textile is a prime example of how design research into wearable technology together with the engineering expertise on in this case capacitive sensing can create new smart textiles with multi-functional capabilities like presence and pressure. 
\begin{marginfigure}
\begin{minipage}{\marginparwidth}
\centering
\includegraphics[trim={0 0 0 -11.5cm},clip,width=0.9\columnwidth]{figures/workshop}
 \caption{PrePre workshop testing materials and configurations at the TU/e E-Lab.}~\label{fig:workshop}
\end{minipage}
\end{marginfigure}
PrePre presents a design collaboration between /d.search labs and Wearable Senses lab at TU/e Industrial Design to create an e-textile and its supporting code to sense pressure and presence on as many as four sensors simultaneously. This collaborative process was selected for a pair of workshops at the Ultra Personalized Smart Textiles (UPST) project at the University of Technology at Eindhoven in part as an ambassador action of the ArcInTexETN Horizon 2020 project\footnote{\url{http://www.ArcInTexETN.eu}}. These workshops explored iterations of touch and presence technologies with interaction designers where new frontiers of sensing and actuating were explored.  

\section{Design}
The capability of sensing does not need to be reduced to pressing against a textile. Knowing about when someone is approaching a textile can add new possibilities on how to interact with it and new dimensions to the design of those interactions. A sensor made with conductive, resistive and insulating materials  intended for "on the body" uses could also be applied for "near the body" uses where softness and tactility are highly valued product features.
\subsection{Design Concept}
Touch is important in interaction, but vicinity is often revealing of behavior and motivations. Not only can vicinity detect hesitation or reluctance in touching, but vicinity can also reveal choosing not to touch. This becomes very interesting when deployed in a garment worn on the body but it is not easily to achieve. An optional shielding layer allows for the use of PrePre on the body, making it only sensitive to one side. Following an iterative design process, and several workshops at the TU/e Wearable Senses Lab (see figure  \ref{fig:workshop}) not only choices of textiles were perfected, but the technique of capacitative sensing was tailored as well to on-the-body and close-to-the-body applications. At the same time the aesthetic qualities were considered as the PrePre is intended for fashion. The PrePre sample is novel in its dual nature of sensing pressure and presence at the same time. Since it requires no extra hardware or fabric layers, garments can be thinner, more flexible and breathable as well as lower in cost and with lower impact on environment. 
\subsection{Relevancy}
 PrePre could be integrated into garments, accessories, furniture, automotive and other places were human computer interaction could be of additional value. The sensor is designed to the human body and the scale of the human hand. The soft and flexible nature of the e-textile sensor allows for its implementation on a multitude of surfaces that the hand interacts with, including clothing and accessories. 
\subsection{Textiles}
The construction of PrePre consisted of a series of conductive, resistive and insulating layers. A low density ESD Foam\footnote{\url{http://nl.farnell.com/multicomp/039-0050/low-density-foam-305x305x6mm/dp/1687866}} was chosen for the resistive layer for its spacer fabric qualities. The low density aspect causes the foam to lift back up quickly after releasing which helps mitigate hysteresis. A conductive silver coated e- textile knit Dorlastan was chosen for the electrode layers due to its soft yet highly conductive feature. The stretch version was chosen to once again aid in resiliency which helps the sensor return to its original state. A hydrophobic polyester was chosen for the insulating layers to protect the conductive layers and prevent influence from humidity. A conductive ripstop nylon was selected for the shielding layer for its conductive conformity. An overview of the stackup of the different materials is shown in Figure \ref{fig:stackup}.

% \begin{marginfigure}
% \begin{minipage}{\marginparwidth}
% \centering
% \includegraphics[width=0.9\columnwidth]{figures/stackup}
% \caption{Stackup for pressure and presence sensor with
% shield}~\label{fig:stackup}
% \end{minipage}
% \end{marginfigure}

\begin{figure}[!htbp]
\centering
  \includegraphics[trim={0 2cm 0 2cm},clip,width=0.9\columnwidth]{figures/stackup}
  \caption{Stackup for pressure and presence sensor with
shield}~\label{fig:stackup}
\end{figure}


\section{Technical Aspects}
\subsection{Resistive Pressure Sensors}
The top three layers in Figure \ref{fig:stackup} show a typical stackup for 
low cost pressure sensors made with electrodes and resistive material. The
resistive material is usually made of a carbon impregnated polymer
with a structure that allows it to be compressed. The material
can be conceptualized as having many parallel resistors. When the material is
compressed some of the resistors will be partially short-circuited due to
non-linear elastic deformations. The partial short-circuits result in a lower
overall resistance of the structure. This is visualized
in Figure \ref{fig:pressure_sensor}.

% \begin{marginfigure}
% \begin{minipage}{\marginparwidth}
\begin{figure}[h!]
\centering
\includegraphics[trim={0 1.4cm 0 1.7cm},clip,width=0.9\columnwidth]{figures/resistive_sensor}
 \caption{Conceptual model of a pressure sensor. Compression causes
  partial short-circuits lowering the overall resistance.}~\label{fig:pressure_sensor}
\end{figure}
% \end{minipage}
% \end{marginfigure}


\subsection{Capacitive Touch / Distance Sensors}
Capacitive sensors are popular sensors in embedded computing due to their low cost and capabilities of detecting approaching human body parts. The parallel plate capacitor model provides a good intuition to the physics behind sensors that measure self-capacitance. In this model one plate of the capacitor is formed by the sensor and the other plate is formed by nearby grounded objects such as a hand or finger. The capacitance is a function of area and distance. There are many different methods to measure self-capacitance. A very popular one will be briefly discussed first followed by a detailed description of a more advanced one used in PrePre.

\subsubsection{RC Charge Method}
A very popular method in the Arduino community is the RC charge
method. In this method the charge and discharge times of a resistor-capacitor
combination is measured. Since the resistor is fixed value, this is a measure
for the value of the capacitor. A detailed description can be found in
\cite{ST2009}.



An intrinsic feature of capacitive touch sensors is that the electric field needs to fringe out of the object to be able to sense the human body. Due to this fringing, the electric field is also easily disturbed by other electric fields or nearby grounded objects such as power lines, electronic devices or metal structures. The slow measurement method of the RC charge makes it more difficult to filter out these disturbances, leading to poor performance of the sensor and poor experiences of capacitive touch for users of the objects. 

\begin{marginfigure}
\begin{minipage}{\marginparwidth}
\centering
\includegraphics[width=0.75\columnwidth]{figures/cap_res_setup_res}
\caption{Resistive pressure sensor used in capacitive and resistive setup in
resistive sensing mode. Grey items are internal to the
microcontroller.}~\label{fig:cap_res_setup_res}
\end{minipage}
\end{marginfigure}

\begin{marginfigure}
\begin{minipage}{\marginparwidth}
\centering
\includegraphics[width=0.75\columnwidth]{figures/cap_res_setup_cap}
\caption{Resistive pressure sensor used in capacitive and resistive setup in
capacitive sensing mode. Grey items are internal to the
microcontroller.}~\label{fig:cap_res_setup_cap}
\end{minipage}
\end{marginfigure}


\subsubsection{CVD Method}
Capacitive Voltage Division (CVD) is a well-known method for self-capacitance does not rely on RC charge times
but instead is based on charge distribution between the sample and hold capacitor
of an ADC ($\textrm{C}_{\textrm{S\&H}}$) and the capacitive sensor. Davison \cite{Davison2013a} gives a
good overview of this method.

CVD does not require any external resistor and the sensor plate is directly connected to an analog input pin.  The microcontroller unit (MCU) starts a
measurement by configuring this pin as a digital output and making this output low,
thereby discharging the sensor. Next, the MCU connects the 
ADC to its supply voltage, which charges $\textrm{C}_{\textrm{S\&H}}$.
Then the sensor pin needs to be reconfigured as analog input and the
multiplexer of the ADC needs to be switched to this input. This will
redistribute the charge on $\textrm{C}_{\textrm{S\&H}}$ over both
$\textrm{C}_{\textrm{S\&H}}$ and the sensor. A larger sensor
capacitance will result in a lower voltage on the sensor and the last step of
this method is to measure this voltage using the ADC.

Since this method does not rely on RC discharge times but uses 
a much faster and higher resolution ADC, more filtering can be applied to the signal to remove
disturbance of other nearby objects and electronic devices. This results in
superior performance and better user experience.

\subsection{Resistive Pressure Sensor as Capacitive Distance Sensor}
The CVD method connects the sensor directly to the analog input of a
MCU, similar to the resistive sensor setup. The resistive sensor can
now be used to also measure capacitance by connecting the other electrode of the
sensor to a GPIO pin instead of ground. This setup is shown in Figures
\ref{fig:cap_res_setup_res} and \ref{fig:cap_res_setup_cap}. Figure \ref{fig:cap_res_setup_res} shows the setup in resistive sensing mode,
which is a standard resistive divider setup using an internal pull resistor
as reference resistor and a digital output pin as ground to complete the
circuit. Figure \ref{fig:cap_res_setup_cap} shows the same circuit but with the GPIO pins
reconfigured for capacitive sensing. In this setup, the internal pull up
resistor is not used and the top electrode of the pressure sensor is only
connected to the analog input. The bottom electrode is connected to a pin 
configured as digital input. This way the sensor is
floating, which is exactly the setup that is needed for the CVD method.

Modern MCUs have fast ADCs (typically 1 MHz or more), allowing it to rapidly switch between
resistive and capacitive sensing modes, using the same sensor for both
resistive pressure and capacitive presence sensing.

\section{Software Features}
In many resistive pressure sensing or capacitive presence sensing applications relative measurements are sufficient. In such cases, a state machine which tracks any background variations on the signal is a simple and effective method to reduce noise. In our case, both the resistive and capacitive signals use the following state machine where each signal has its own instance and parameter settings. 

Similar to the state machine described by Bohn \cite{Bohn2009}, the state
machine for each sensor can be in five states: Calibrating, Released, Released
to Pressed, Pressed and Pressed to Released. This is shown in Figure
\ref{fig:state_machine}.

In the Calibrating and Released states, the background variations are tracked
using an exponential decaying filter and stored in average $a$. In all other
states the variations are not tracked. In the Released state, if the
most recent measurement $x$ is more than $P$ counts below average $a$, the state
is changed to Released to Pressed. If in the next measurement $x$ is less
than $P$ counts below the average, the state is changed back to Released. If
however the next $C$ measurements are all more than $P$ counts below this
average, the state is changed to Pressed.

% \begin{marginfigure}
% \begin{minipage}{\marginparwidth}
\begin{figure}
\centering
\includegraphics[trim={0 6.2cm 0 6.8cm},clip,width=0.9\columnwidth]{figures/state_machine}
 \caption{State machine for resistive and capacitive measurements to track
background variations.}~\label{fig:state_machine}
\end{figure}
% \end{minipage}
% \end{marginfigure}

Similarly the state moves from the Pressed state to the Pressed to Released
state and from Pressed to Released to the Released state. This is shown in Figure \ref{fig:state_machine}.

By changing parameters $N$, $C$, $P$ and $R$ the amount of filtering and
speed of detection can be tuned to the application. Once tuned
properly, the difference of $x$ and $a$ is a measure for how close a user is to
the sensor (in capacitive mode) or for how much pressure a user applies
to the sensor (in resistive mode).

This state machine and all code for resistive and capacitive sensing are
implemented on an Arduino and made available freely (BSD licensed) at
GitHub\footnote{\url{https://github.com/admarschoonen/resistive_cap_touch}}.

\section{Shielding}
In capacitive mode, the sensor is also sensitive on the underside. If
the distance between the underside of the sensor and the human body is
relatively constant, tuning could be sufficient to filter
this out and make the sensor only sensitive to large and / or rapid variations.

% \begin{marginfigure}
% \begin{minipage}{\marginparwidth}
\begin{figure}
\centering
\includegraphics[width=0.9\columnwidth]{figures/shield_circuit}
  \caption{Circuit to shield underside of capacitive
sensor.}~\label{fig:shield_circuit}
\end{figure}
% \end{minipage}
% \end{marginfigure}

However, some applications such as lose fitting clothing require more filtering
and benefit from a shield underneath the
sensor. Connecting this shield to ground effectively removes all of the
capacitance variation but also reduces the sensitivity of the sensor. By
connecting the sensor to the input of a 1 x opamp and the output of the
opamp to the shield, the voltage of the shield is always close to the
voltage on the sensor. This is similar to what Davison describes in
\cite{Davison2013a}. The electric field underneath the sensor is then
virtually zero and no sensitivity is lost. Note that in many applications also
the cable to the sensor should be shielded. A schematic diagram
is shown in Figure \ref{fig:shield_circuit}.




% \section{Mechanical Features}
% An overview of the stackup of the total sensor is shown in Figure
% \ref{fig:stackup}. In this figure, the electrode and shield material can be
% conductive textile, the insulating material can be any non-compressable
% insulating material (for example cotton) and the pressure sensitive material can
% be Velostat or ESD foam.


\section{Conclusion}
The possibilities of a reliable, textile, soft presence and pressure sensor are numerous: from jacket sleeves that can detect swipes and touches that control your cell phone to shoes that sense your steps and stride of both feet. The benefits of a sensor that can see a users hands as it moves over their body are numerous. This is built upon the reliability detailed in the combination of capacitive distance sensing using the CVD method for presence sensing with resistive sensing for pressure sensing. The robustness of the CVD method as well as the required circuit and MCU features make it ideal to combine with existing resistive pressure sensing applications to enhance the user experience by not only sensing how hard a user presses on a button but already giving feedback to the user when approaching the button. The choice of textiles increases the reliability and performance of the sensor in their specific contexts. In the use of shoes we measure not only how hard the foot is being pressed, but if and how far the foot is disconnecting from the shoe or the presence of the other foot. New aspects of arch lift and stride detection are possible. In garments, beyond seeing a single users body movements, we can use the sensors to understand embodied interaction with multiple people as they approach and touch each other in performance or everyday activities. In the context of an automative steering wheel, we can understand not only where someone is touching the wheel, but how they move their hands when engaged in an maneuver such as turning a corner. The code already provided allows for multiple sensors to generate movement vectors along side specific specific touch location. Increasing sensor density is a serious possibility that the authors intend to explore further. 

\section{Acknowledgements}
The authors would like to thank the Wearable Senses Lab, /dSearch Lab and E-Labs
along with the Designing Quality in Interaction group of the Industrial Design
department of the Eindhoven University of Technology. Support for this project
also comes from the Marie Curie Horizon 2020 Action ArcInTexETN. Special thanks
to Dr. Oscar Tomico and Dr. Stephan Wensveen of TU Eindhoven.

% BALANCE COLUMNS
\balance{} 
% REFERENCES FORMAT
% References must be the same font size as other body text.
\bibliographystyle{SIGCHI-Reference-Format}
\bibliography{prepre}

\end{document}

%%% Local Variables:
%%% mode: latex
%%% TeX-master: t
%%% End:
